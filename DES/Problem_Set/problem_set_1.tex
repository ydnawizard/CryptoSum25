\documentclass{article}

\usepackage{fancyhdr}
\usepackage{extramarks}
\usepackage{amsmath}
\usepackage{amsthm}
\usepackage{amsfonts}
\usepackage{tikz}
\usepackage[plain]{algorithm}
\usepackage{algpseudocode}

\usetikzlibrary{automata,positioning}

%
% Basic Document Settings
%

\topmargin=-0.45in
\evensidemargin=0in
\oddsidemargin=0in
\textwidth=6.5in
\textheight=9.0in
\headsep=0.25in


\linespread{1.1}

\pagestyle{fancy}
\lhead{\hmwkAuthorName}
\chead{\hmwkClass \text{, }\hmwkClassInstructor\:\text{, }\hmwkTitle}
\rhead{\firstxmark}
\lfoot{\lastxmark}
\cfoot{\thepage}

\renewcommand\headrulewidth{0.4pt}
\renewcommand\footrulewidth{0.4pt}

\setlength\parindent{0pt}

%
% Create Problem Sections
%

\newcommand{\enterProblemHeader}[1]{
    \nobreak\extramarks{}{Problem \arabic{#1} continued on next page\ldots}\nobreak{}
    \nobreak\extramarks{Problem \arabic{#1} (continued)}{Problem \arabic{#1} continued on next page\ldots}\nobreak{}
}

\newcommand{\exitProblemHeader}[1]{
    \nobreak\extramarks{Problem \arabic{#1} (continued)}{Problem \arabic{#1} continued on next page\ldots}\nobreak{}
    \stepcounter{#1}
    \nobreak\extramarks{Problem \arabic{#1}}{}\nobreak{}
}

\setcounter{secnumdepth}{0}
\newcounter{partCounter}
\newcounter{homeworkProblemCounter}
\setcounter{homeworkProblemCounter}{1}
\nobreak\extramarks{Problem \arabic{homeworkProblemCounter}}{}\nobreak{}

%
% Homework Problem Environment
%
% This environment takes an optional argument. When given, it will adjust the
% problem counter. This is useful for when the problems given for your
% assignment aren't sequential. See the last 3 problems of this template for an
% example.
%
\newenvironment{homeworkProblem}[1][-1]{
    \ifnum#1>0
        \setcounter{homeworkProblemCounter}{#1}
    \fi
    \section{Problem \arabic{homeworkProblemCounter}}
    \setcounter{partCounter}{1}
    \enterProblemHeader{homeworkProblemCounter}
}{
    \exitProblemHeader{homeworkProblemCounter}
}

%
% Homework Details
%   - Title
%   - Due date
%   - Class
%   - Section/Time
%   - Instructor
%   - Author
%

\newcommand{\hmwkTitle}{Problem Set\ \#1}
\newcommand{\hmwkDueDate}{Monday, June 2nd}
\newcommand{\hmwkClass}{Cryptography}
\newcommand{\hmwkClassInstructor}{Elena Machkasova}
\newcommand{\hmwkAuthorName}{\textbf{Ellis Weglewski}}

%
% Title Page
%

\title{
    \vspace{2in}
    \textmd{\textbf{\hmwkClass:\ \hmwkTitle}}\\
    \normalsize\vspace{0.1in}\small{Due\ on\ \hmwkDueDate\ at 10:00pm}\\
    \vspace{0.1in}\large{\textit{\hmwkClassInstructor}}
    \vspace{3in}
}

\author{\hmwkAuthorName}
\date{}

\renewcommand{\part}[1]{\textbf{\large Part \Alph{partCounter}}\stepcounter{partCounter}\\}

%
% Various Helper Commands
%

% Useful for algorithms
\newcommand{\alg}[1]{\textsc{\bfseries \footnotesize #1}}

% For derivatives
\newcommand{\deriv}[1]{\frac{\mathrm{d}}{\mathrm{d}x} (#1)}

% For partial derivatives
\newcommand{\pderiv}[2]{\frac{\partial}{\partial #1} (#2)}

% Integral dx
\newcommand{\dx}{\mathrm{d}x}

% Alias for the Solution section header
\newcommand{\solution}{\textbf{\large Solution}}

% Probability commands: Expectation, Variance, Covariance, Bias
\newcommand{\E}{\mathrm{E}}
\newcommand{\Var}{\mathrm{Var}}
\newcommand{\Cov}{\mathrm{Cov}}
\newcommand{\Bias}{\mathrm{Bias}}

\begin{document}

\maketitle

\pagebreak

\begin{homeworkProblem}
	\textbf{(3.1) Question:}\\
	Show that $S_1(x_1)\oplus S_1(x_2)\neq S_1(x_1\oplus x_2)$ for:
	\begin{enumerate}
		\item \(x_1 = 000000, x_2 = 000001\)
		\item \(x_1 = 111111, x_2 = 100000\)
		\item \(x_1 = 101010, x_2 = 010101\)
	\end{enumerate}
	\textbf{Solutions:}\\
	\begin{enumerate}
		\item \begin{enumerate}
				\item \((S_1((x_1= 0,0 = 14) = 1110 \oplus S_1((x_2=0,1 = 00 = 0000)) = 1110\)
				\item \(S_1(000000\oplus 0000001) = S_1(000001) = 0000\)
				\item \(0000\neq 1110\)
			\end{enumerate}
		\item \begin{enumerate}
				\item \((S_1(x_1 = 15,3 = 13) = 1101 \oplus S_1(x_2 = 0,2 = 04)=0100) = 1001\)
				\item \(S_1(111111\oplus 100000) = S_1(011111) = 1000\)
				\item \(1001\neq1000\)
			\end{enumerate}
		\item \begin{enumerate}
				\item \((S_1(x_1 = 5,2 = 06) = 0110 \oplus S_1(x_2 = 10,1 = 12) = 1100) = 1010\)
				\item \(S_1(101010\oplus 010101) = S_1(111111) = 1101\)
				\item \(1010\neq 1101\)
			\end{enumerate}
	\end{enumerate}
\end{homeworkProblem}

\begin{homeworkProblem}
	\textbf{(3.2) Question:}\\
	We want to verify that $IP(\cdot)$ and $IP^{-1}(\cdot)$ are truly inverse operations. We consider a vector $x=(x_1,x_2,...,x_{64})$ of 64 bit. Show that $IP^{-1}(IP(x))=x$ for the first five
	bits of $x$,i.e. for $x_i = 1,2,3,4,5$.\\
	\textbf{Solution:}\\
	Via pg. 70: $IP(Y) = IP(IP^{-1}(R_{16}L_{16}))$. I take this to imply: $IP^{-1}(Y)=IP^{-1}(IP(R_{16}L_{16}))$.
	If we look at $IP$ and $IP^{-1}$ boxes on pg 62, we can see where each byte is sent after each operation.
	The byte in position 1 is sent to position 40 after $IP$ and the byte in position 40 is sent to position 1 after $IP^{-1}$.
	This means that $IP^{-1}$ is undoing what $IP$ did which makes them mutually inverse.
	We can construct a flow chart to illustrate where each byte goes in each step of the operation $IP^{-1}(IP(x))=x$:
	\begin{enumerate}
		\item \(x_1\rightarrow x_{40}\rightarrow x_1\)
		\item \(x_2\rightarrow x_8\rightarrow x_2\)
		\item \(x_3\rightarrow x_{48}\rightarrow x_3\)
		\item \(x_4\rightarrow x_{16}\rightarrow x_4\)
		\item \(x_5\rightarrow x_{56}\rightarrow x_5\)
	\end{enumerate}

\end{homeworkProblem}

\pagebreak

\begin{homeworkProblem}
	\textbf{(3.3) Question:}\\
	What is the output of the first round of the DES algorithm when the plaintext and the key are both all zeros?\\
	/textbf{Solution:}\\


\end{homeworkProblem}

\end{document}
